\documentclass[a4paper]{article}
\usepackage{amsmath}

\title{The Error Function}
\author{Markus R. Mosbech}
\date{}

\begin{document}
\maketitle

\begin{abstract}
In mathematics, the error function (also called the Gauss error function) is a
special function (non-elementary) of sigmoid shape that occurs in probability,
statistics, and partial differential equations describing diffusion.
It is defined as:
$\mathrm{erf}(x) = \frac{2}{\sqrt{\pi}}\int_0^\infty e^{-t^2} \, dt$.
In statistics, for nonnegative values of x, the error function has the following
interpretation: for a random variable Y that is normally distributed with mean 0
and variance 1/2, erf(x) describes the probability of Y falling in the range
[-x, x].
\end{abstract}

\section{The name "error function"}
The name and abbreviation for the error function (and the error function
complement) were developed by J. W. L. Glaisher in 1871 on account of its
connection with "the theory of Probability, and notably the theory of Errors."

\section{Properties}
The property $\mathrm{erf} ⁡ ( - z ) = − \mathrm{erf} ⁡ ( z )$  means that the
error function is an odd function. This directly results from the fact that the
integrand $e^{ - t^2}$ is an even function.

For any complex number $z$:
\begin{equation}
  \mathrm{erf}(\bar{z})=\bar{\mathrm{erf}(z)}
\end{equation}
where $\bar{z}$ is the complex conjugate of $z$.

The error function at $+\infty$ is exactly 1 (see Gaussian integral). At the
real axis, $\mathrm{erf}(z)$ approaches unity at $ z \rightarrow \infty$ and
$-1$ at $z \rightarrow -\infty$. At the imaginary axis, it tends to $\pm i\infty$.

The error function is an entire function; it has no singularities (except that at
infinity) and its Taylor expansion always converges.

The shape of the error function can be seen in Figure~\ref{fig:erf}.
\begin{figure}
  % GNUPLOT: LaTeX picture with Postscript
\begingroup%
\makeatletter%
\newcommand{\GNUPLOTspecial}{%
  \@sanitize\catcode`\%=14\relax\special}%
\setlength{\unitlength}{0.0500bp}%
\begin{picture}(7200,5040)(0,0)%
  {\GNUPLOTspecial{"
%!PS-Adobe-2.0 EPSF-2.0
%%Title: plot.tex
%%Creator: gnuplot 5.0 patchlevel 7
%%CreationDate: Sun May  6 16:49:27 2018
%%DocumentFonts: 
%%BoundingBox: 0 0 360 252
%%EndComments
%%BeginProlog
/gnudict 256 dict def
gnudict begin
%
% The following true/false flags may be edited by hand if desired.
% The unit line width and grayscale image gamma correction may also be changed.
%
/Color false def
/Blacktext true def
/Solid false def
/Dashlength 1 def
/Landscape false def
/Level1 false def
/Level3 false def
/Rounded false def
/ClipToBoundingBox false def
/SuppressPDFMark false def
/TransparentPatterns false def
/gnulinewidth 5.000 def
/userlinewidth gnulinewidth def
/Gamma 1.0 def
/BackgroundColor {-1.000 -1.000 -1.000} def
%
/vshift -66 def
/dl1 {
  10.0 Dashlength userlinewidth gnulinewidth div mul mul mul
  Rounded { currentlinewidth 0.75 mul sub dup 0 le { pop 0.01 } if } if
} def
/dl2 {
  10.0 Dashlength userlinewidth gnulinewidth div mul mul mul
  Rounded { currentlinewidth 0.75 mul add } if
} def
/hpt_ 31.5 def
/vpt_ 31.5 def
/hpt hpt_ def
/vpt vpt_ def
/doclip {
  ClipToBoundingBox {
    newpath 0 0 moveto 360 0 lineto 360 252 lineto 0 252 lineto closepath
    clip
  } if
} def
%
% Gnuplot Prolog Version 5.1 (Oct 2015)
%
%/SuppressPDFMark true def
%
/M {moveto} bind def
/L {lineto} bind def
/R {rmoveto} bind def
/V {rlineto} bind def
/N {newpath moveto} bind def
/Z {closepath} bind def
/C {setrgbcolor} bind def
/f {rlineto fill} bind def
/g {setgray} bind def
/Gshow {show} def   % May be redefined later in the file to support UTF-8
/vpt2 vpt 2 mul def
/hpt2 hpt 2 mul def
/Lshow {currentpoint stroke M 0 vshift R 
	Blacktext {gsave 0 setgray textshow grestore} {textshow} ifelse} def
/Rshow {currentpoint stroke M dup stringwidth pop neg vshift R
	Blacktext {gsave 0 setgray textshow grestore} {textshow} ifelse} def
/Cshow {currentpoint stroke M dup stringwidth pop -2 div vshift R 
	Blacktext {gsave 0 setgray textshow grestore} {textshow} ifelse} def
/UP {dup vpt_ mul /vpt exch def hpt_ mul /hpt exch def
  /hpt2 hpt 2 mul def /vpt2 vpt 2 mul def} def
/DL {Color {setrgbcolor Solid {pop []} if 0 setdash}
 {pop pop pop 0 setgray Solid {pop []} if 0 setdash} ifelse} def
/BL {stroke userlinewidth 2 mul setlinewidth
	Rounded {1 setlinejoin 1 setlinecap} if} def
/AL {stroke userlinewidth 2 div setlinewidth
	Rounded {1 setlinejoin 1 setlinecap} if} def
/UL {dup gnulinewidth mul /userlinewidth exch def
	dup 1 lt {pop 1} if 10 mul /udl exch def} def
/PL {stroke userlinewidth setlinewidth
	Rounded {1 setlinejoin 1 setlinecap} if} def
3.8 setmiterlimit
% Classic Line colors (version 5.0)
/LCw {1 1 1} def
/LCb {0 0 0} def
/LCa {0 0 0} def
/LC0 {1 0 0} def
/LC1 {0 1 0} def
/LC2 {0 0 1} def
/LC3 {1 0 1} def
/LC4 {0 1 1} def
/LC5 {1 1 0} def
/LC6 {0 0 0} def
/LC7 {1 0.3 0} def
/LC8 {0.5 0.5 0.5} def
% Default dash patterns (version 5.0)
/LTB {BL [] LCb DL} def
/LTw {PL [] 1 setgray} def
/LTb {PL [] LCb DL} def
/LTa {AL [1 udl mul 2 udl mul] 0 setdash LCa setrgbcolor} def
/LT0 {PL [] LC0 DL} def
/LT1 {PL [2 dl1 3 dl2] LC1 DL} def
/LT2 {PL [1 dl1 1.5 dl2] LC2 DL} def
/LT3 {PL [6 dl1 2 dl2 1 dl1 2 dl2] LC3 DL} def
/LT4 {PL [1 dl1 2 dl2 6 dl1 2 dl2 1 dl1 2 dl2] LC4 DL} def
/LT5 {PL [4 dl1 2 dl2] LC5 DL} def
/LT6 {PL [1.5 dl1 1.5 dl2 1.5 dl1 1.5 dl2 1.5 dl1 6 dl2] LC6 DL} def
/LT7 {PL [3 dl1 3 dl2 1 dl1 3 dl2] LC7 DL} def
/LT8 {PL [2 dl1 2 dl2 2 dl1 6 dl2] LC8 DL} def
/SL {[] 0 setdash} def
/Pnt {stroke [] 0 setdash gsave 1 setlinecap M 0 0 V stroke grestore} def
/Dia {stroke [] 0 setdash 2 copy vpt add M
  hpt neg vpt neg V hpt vpt neg V
  hpt vpt V hpt neg vpt V closepath stroke
  Pnt} def
/Pls {stroke [] 0 setdash vpt sub M 0 vpt2 V
  currentpoint stroke M
  hpt neg vpt neg R hpt2 0 V stroke
 } def
/Box {stroke [] 0 setdash 2 copy exch hpt sub exch vpt add M
  0 vpt2 neg V hpt2 0 V 0 vpt2 V
  hpt2 neg 0 V closepath stroke
  Pnt} def
/Crs {stroke [] 0 setdash exch hpt sub exch vpt add M
  hpt2 vpt2 neg V currentpoint stroke M
  hpt2 neg 0 R hpt2 vpt2 V stroke} def
/TriU {stroke [] 0 setdash 2 copy vpt 1.12 mul add M
  hpt neg vpt -1.62 mul V
  hpt 2 mul 0 V
  hpt neg vpt 1.62 mul V closepath stroke
  Pnt} def
/Star {2 copy Pls Crs} def
/BoxF {stroke [] 0 setdash exch hpt sub exch vpt add M
  0 vpt2 neg V hpt2 0 V 0 vpt2 V
  hpt2 neg 0 V closepath fill} def
/TriUF {stroke [] 0 setdash vpt 1.12 mul add M
  hpt neg vpt -1.62 mul V
  hpt 2 mul 0 V
  hpt neg vpt 1.62 mul V closepath fill} def
/TriD {stroke [] 0 setdash 2 copy vpt 1.12 mul sub M
  hpt neg vpt 1.62 mul V
  hpt 2 mul 0 V
  hpt neg vpt -1.62 mul V closepath stroke
  Pnt} def
/TriDF {stroke [] 0 setdash vpt 1.12 mul sub M
  hpt neg vpt 1.62 mul V
  hpt 2 mul 0 V
  hpt neg vpt -1.62 mul V closepath fill} def
/DiaF {stroke [] 0 setdash vpt add M
  hpt neg vpt neg V hpt vpt neg V
  hpt vpt V hpt neg vpt V closepath fill} def
/Pent {stroke [] 0 setdash 2 copy gsave
  translate 0 hpt M 4 {72 rotate 0 hpt L} repeat
  closepath stroke grestore Pnt} def
/PentF {stroke [] 0 setdash gsave
  translate 0 hpt M 4 {72 rotate 0 hpt L} repeat
  closepath fill grestore} def
/Circle {stroke [] 0 setdash 2 copy
  hpt 0 360 arc stroke Pnt} def
/CircleF {stroke [] 0 setdash hpt 0 360 arc fill} def
/C0 {BL [] 0 setdash 2 copy moveto vpt 90 450 arc} bind def
/C1 {BL [] 0 setdash 2 copy moveto
	2 copy vpt 0 90 arc closepath fill
	vpt 0 360 arc closepath} bind def
/C2 {BL [] 0 setdash 2 copy moveto
	2 copy vpt 90 180 arc closepath fill
	vpt 0 360 arc closepath} bind def
/C3 {BL [] 0 setdash 2 copy moveto
	2 copy vpt 0 180 arc closepath fill
	vpt 0 360 arc closepath} bind def
/C4 {BL [] 0 setdash 2 copy moveto
	2 copy vpt 180 270 arc closepath fill
	vpt 0 360 arc closepath} bind def
/C5 {BL [] 0 setdash 2 copy moveto
	2 copy vpt 0 90 arc
	2 copy moveto
	2 copy vpt 180 270 arc closepath fill
	vpt 0 360 arc} bind def
/C6 {BL [] 0 setdash 2 copy moveto
	2 copy vpt 90 270 arc closepath fill
	vpt 0 360 arc closepath} bind def
/C7 {BL [] 0 setdash 2 copy moveto
	2 copy vpt 0 270 arc closepath fill
	vpt 0 360 arc closepath} bind def
/C8 {BL [] 0 setdash 2 copy moveto
	2 copy vpt 270 360 arc closepath fill
	vpt 0 360 arc closepath} bind def
/C9 {BL [] 0 setdash 2 copy moveto
	2 copy vpt 270 450 arc closepath fill
	vpt 0 360 arc closepath} bind def
/C10 {BL [] 0 setdash 2 copy 2 copy moveto vpt 270 360 arc closepath fill
	2 copy moveto
	2 copy vpt 90 180 arc closepath fill
	vpt 0 360 arc closepath} bind def
/C11 {BL [] 0 setdash 2 copy moveto
	2 copy vpt 0 180 arc closepath fill
	2 copy moveto
	2 copy vpt 270 360 arc closepath fill
	vpt 0 360 arc closepath} bind def
/C12 {BL [] 0 setdash 2 copy moveto
	2 copy vpt 180 360 arc closepath fill
	vpt 0 360 arc closepath} bind def
/C13 {BL [] 0 setdash 2 copy moveto
	2 copy vpt 0 90 arc closepath fill
	2 copy moveto
	2 copy vpt 180 360 arc closepath fill
	vpt 0 360 arc closepath} bind def
/C14 {BL [] 0 setdash 2 copy moveto
	2 copy vpt 90 360 arc closepath fill
	vpt 0 360 arc} bind def
/C15 {BL [] 0 setdash 2 copy vpt 0 360 arc closepath fill
	vpt 0 360 arc closepath} bind def
/Rec {newpath 4 2 roll moveto 1 index 0 rlineto 0 exch rlineto
	neg 0 rlineto closepath} bind def
/Square {dup Rec} bind def
/Bsquare {vpt sub exch vpt sub exch vpt2 Square} bind def
/S0 {BL [] 0 setdash 2 copy moveto 0 vpt rlineto BL Bsquare} bind def
/S1 {BL [] 0 setdash 2 copy vpt Square fill Bsquare} bind def
/S2 {BL [] 0 setdash 2 copy exch vpt sub exch vpt Square fill Bsquare} bind def
/S3 {BL [] 0 setdash 2 copy exch vpt sub exch vpt2 vpt Rec fill Bsquare} bind def
/S4 {BL [] 0 setdash 2 copy exch vpt sub exch vpt sub vpt Square fill Bsquare} bind def
/S5 {BL [] 0 setdash 2 copy 2 copy vpt Square fill
	exch vpt sub exch vpt sub vpt Square fill Bsquare} bind def
/S6 {BL [] 0 setdash 2 copy exch vpt sub exch vpt sub vpt vpt2 Rec fill Bsquare} bind def
/S7 {BL [] 0 setdash 2 copy exch vpt sub exch vpt sub vpt vpt2 Rec fill
	2 copy vpt Square fill Bsquare} bind def
/S8 {BL [] 0 setdash 2 copy vpt sub vpt Square fill Bsquare} bind def
/S9 {BL [] 0 setdash 2 copy vpt sub vpt vpt2 Rec fill Bsquare} bind def
/S10 {BL [] 0 setdash 2 copy vpt sub vpt Square fill 2 copy exch vpt sub exch vpt Square fill
	Bsquare} bind def
/S11 {BL [] 0 setdash 2 copy vpt sub vpt Square fill 2 copy exch vpt sub exch vpt2 vpt Rec fill
	Bsquare} bind def
/S12 {BL [] 0 setdash 2 copy exch vpt sub exch vpt sub vpt2 vpt Rec fill Bsquare} bind def
/S13 {BL [] 0 setdash 2 copy exch vpt sub exch vpt sub vpt2 vpt Rec fill
	2 copy vpt Square fill Bsquare} bind def
/S14 {BL [] 0 setdash 2 copy exch vpt sub exch vpt sub vpt2 vpt Rec fill
	2 copy exch vpt sub exch vpt Square fill Bsquare} bind def
/S15 {BL [] 0 setdash 2 copy Bsquare fill Bsquare} bind def
/D0 {gsave translate 45 rotate 0 0 S0 stroke grestore} bind def
/D1 {gsave translate 45 rotate 0 0 S1 stroke grestore} bind def
/D2 {gsave translate 45 rotate 0 0 S2 stroke grestore} bind def
/D3 {gsave translate 45 rotate 0 0 S3 stroke grestore} bind def
/D4 {gsave translate 45 rotate 0 0 S4 stroke grestore} bind def
/D5 {gsave translate 45 rotate 0 0 S5 stroke grestore} bind def
/D6 {gsave translate 45 rotate 0 0 S6 stroke grestore} bind def
/D7 {gsave translate 45 rotate 0 0 S7 stroke grestore} bind def
/D8 {gsave translate 45 rotate 0 0 S8 stroke grestore} bind def
/D9 {gsave translate 45 rotate 0 0 S9 stroke grestore} bind def
/D10 {gsave translate 45 rotate 0 0 S10 stroke grestore} bind def
/D11 {gsave translate 45 rotate 0 0 S11 stroke grestore} bind def
/D12 {gsave translate 45 rotate 0 0 S12 stroke grestore} bind def
/D13 {gsave translate 45 rotate 0 0 S13 stroke grestore} bind def
/D14 {gsave translate 45 rotate 0 0 S14 stroke grestore} bind def
/D15 {gsave translate 45 rotate 0 0 S15 stroke grestore} bind def
/DiaE {stroke [] 0 setdash vpt add M
  hpt neg vpt neg V hpt vpt neg V
  hpt vpt V hpt neg vpt V closepath stroke} def
/BoxE {stroke [] 0 setdash exch hpt sub exch vpt add M
  0 vpt2 neg V hpt2 0 V 0 vpt2 V
  hpt2 neg 0 V closepath stroke} def
/TriUE {stroke [] 0 setdash vpt 1.12 mul add M
  hpt neg vpt -1.62 mul V
  hpt 2 mul 0 V
  hpt neg vpt 1.62 mul V closepath stroke} def
/TriDE {stroke [] 0 setdash vpt 1.12 mul sub M
  hpt neg vpt 1.62 mul V
  hpt 2 mul 0 V
  hpt neg vpt -1.62 mul V closepath stroke} def
/PentE {stroke [] 0 setdash gsave
  translate 0 hpt M 4 {72 rotate 0 hpt L} repeat
  closepath stroke grestore} def
/CircE {stroke [] 0 setdash 
  hpt 0 360 arc stroke} def
/Opaque {gsave closepath 1 setgray fill grestore 0 setgray closepath} def
/DiaW {stroke [] 0 setdash vpt add M
  hpt neg vpt neg V hpt vpt neg V
  hpt vpt V hpt neg vpt V Opaque stroke} def
/BoxW {stroke [] 0 setdash exch hpt sub exch vpt add M
  0 vpt2 neg V hpt2 0 V 0 vpt2 V
  hpt2 neg 0 V Opaque stroke} def
/TriUW {stroke [] 0 setdash vpt 1.12 mul add M
  hpt neg vpt -1.62 mul V
  hpt 2 mul 0 V
  hpt neg vpt 1.62 mul V Opaque stroke} def
/TriDW {stroke [] 0 setdash vpt 1.12 mul sub M
  hpt neg vpt 1.62 mul V
  hpt 2 mul 0 V
  hpt neg vpt -1.62 mul V Opaque stroke} def
/PentW {stroke [] 0 setdash gsave
  translate 0 hpt M 4 {72 rotate 0 hpt L} repeat
  Opaque stroke grestore} def
/CircW {stroke [] 0 setdash 
  hpt 0 360 arc Opaque stroke} def
/BoxFill {gsave Rec 1 setgray fill grestore} def
/Density {
  /Fillden exch def
  currentrgbcolor
  /ColB exch def /ColG exch def /ColR exch def
  /ColR ColR Fillden mul Fillden sub 1 add def
  /ColG ColG Fillden mul Fillden sub 1 add def
  /ColB ColB Fillden mul Fillden sub 1 add def
  ColR ColG ColB setrgbcolor} def
/BoxColFill {gsave Rec PolyFill} def
/PolyFill {gsave Density fill grestore grestore} def
/h {rlineto rlineto rlineto gsave closepath fill grestore} bind def
%
% PostScript Level 1 Pattern Fill routine for rectangles
% Usage: x y w h s a XX PatternFill
%	x,y = lower left corner of box to be filled
%	w,h = width and height of box
%	  a = angle in degrees between lines and x-axis
%	 XX = 0/1 for no/yes cross-hatch
%
/PatternFill {gsave /PFa [ 9 2 roll ] def
  PFa 0 get PFa 2 get 2 div add PFa 1 get PFa 3 get 2 div add translate
  PFa 2 get -2 div PFa 3 get -2 div PFa 2 get PFa 3 get Rec
  TransparentPatterns {} {gsave 1 setgray fill grestore} ifelse
  clip
  currentlinewidth 0.5 mul setlinewidth
  /PFs PFa 2 get dup mul PFa 3 get dup mul add sqrt def
  0 0 M PFa 5 get rotate PFs -2 div dup translate
  0 1 PFs PFa 4 get div 1 add floor cvi
	{PFa 4 get mul 0 M 0 PFs V} for
  0 PFa 6 get ne {
	0 1 PFs PFa 4 get div 1 add floor cvi
	{PFa 4 get mul 0 2 1 roll M PFs 0 V} for
 } if
  stroke grestore} def
%
/languagelevel where
 {pop languagelevel} {1} ifelse
dup 2 lt
	{/InterpretLevel1 true def
	 /InterpretLevel3 false def}
	{/InterpretLevel1 Level1 def
	 2 gt
	    {/InterpretLevel3 Level3 def}
	    {/InterpretLevel3 false def}
	 ifelse }
 ifelse
%
% PostScript level 2 pattern fill definitions
%
/Level2PatternFill {
/Tile8x8 {/PaintType 2 /PatternType 1 /TilingType 1 /BBox [0 0 8 8] /XStep 8 /YStep 8}
	bind def
/KeepColor {currentrgbcolor [/Pattern /DeviceRGB] setcolorspace} bind def
<< Tile8x8
 /PaintProc {0.5 setlinewidth pop 0 0 M 8 8 L 0 8 M 8 0 L stroke} 
>> matrix makepattern
/Pat1 exch def
<< Tile8x8
 /PaintProc {0.5 setlinewidth pop 0 0 M 8 8 L 0 8 M 8 0 L stroke
	0 4 M 4 8 L 8 4 L 4 0 L 0 4 L stroke}
>> matrix makepattern
/Pat2 exch def
<< Tile8x8
 /PaintProc {0.5 setlinewidth pop 0 0 M 0 8 L
	8 8 L 8 0 L 0 0 L fill}
>> matrix makepattern
/Pat3 exch def
<< Tile8x8
 /PaintProc {0.5 setlinewidth pop -4 8 M 8 -4 L
	0 12 M 12 0 L stroke}
>> matrix makepattern
/Pat4 exch def
<< Tile8x8
 /PaintProc {0.5 setlinewidth pop -4 0 M 8 12 L
	0 -4 M 12 8 L stroke}
>> matrix makepattern
/Pat5 exch def
<< Tile8x8
 /PaintProc {0.5 setlinewidth pop -2 8 M 4 -4 L
	0 12 M 8 -4 L 4 12 M 10 0 L stroke}
>> matrix makepattern
/Pat6 exch def
<< Tile8x8
 /PaintProc {0.5 setlinewidth pop -2 0 M 4 12 L
	0 -4 M 8 12 L 4 -4 M 10 8 L stroke}
>> matrix makepattern
/Pat7 exch def
<< Tile8x8
 /PaintProc {0.5 setlinewidth pop 8 -2 M -4 4 L
	12 0 M -4 8 L 12 4 M 0 10 L stroke}
>> matrix makepattern
/Pat8 exch def
<< Tile8x8
 /PaintProc {0.5 setlinewidth pop 0 -2 M 12 4 L
	-4 0 M 12 8 L -4 4 M 8 10 L stroke}
>> matrix makepattern
/Pat9 exch def
/Pattern1 {PatternBgnd KeepColor Pat1 setpattern} bind def
/Pattern2 {PatternBgnd KeepColor Pat2 setpattern} bind def
/Pattern3 {PatternBgnd KeepColor Pat3 setpattern} bind def
/Pattern4 {PatternBgnd KeepColor Landscape {Pat5} {Pat4} ifelse setpattern} bind def
/Pattern5 {PatternBgnd KeepColor Landscape {Pat4} {Pat5} ifelse setpattern} bind def
/Pattern6 {PatternBgnd KeepColor Landscape {Pat9} {Pat6} ifelse setpattern} bind def
/Pattern7 {PatternBgnd KeepColor Landscape {Pat8} {Pat7} ifelse setpattern} bind def
} def
%
%
%End of PostScript Level 2 code
%
/PatternBgnd {
  TransparentPatterns {} {gsave 1 setgray fill grestore} ifelse
} def
%
% Substitute for Level 2 pattern fill codes with
% grayscale if Level 2 support is not selected.
%
/Level1PatternFill {
/Pattern1 {0.250 Density} bind def
/Pattern2 {0.500 Density} bind def
/Pattern3 {0.750 Density} bind def
/Pattern4 {0.125 Density} bind def
/Pattern5 {0.375 Density} bind def
/Pattern6 {0.625 Density} bind def
/Pattern7 {0.875 Density} bind def
} def
%
% Now test for support of Level 2 code
%
Level1 {Level1PatternFill} {Level2PatternFill} ifelse
%
/Symbol-Oblique /Symbol findfont [1 0 .167 1 0 0] makefont
dup length dict begin {1 index /FID eq {pop pop} {def} ifelse} forall
currentdict end definefont pop
%
Level1 SuppressPDFMark or 
{} {
/SDict 10 dict def
systemdict /pdfmark known not {
  userdict /pdfmark systemdict /cleartomark get put
} if
SDict begin [
  /Title (plot.tex)
  /Subject (gnuplot plot)
  /Creator (gnuplot 5.0 patchlevel 7)
%  /Producer (gnuplot)
%  /Keywords ()
  /CreationDate (Sun May  6 16:49:27 2018)
  /DOCINFO pdfmark
end
} ifelse
%
% Support for boxed text - Ethan A Merritt May 2005
%
/InitTextBox { userdict /TBy2 3 -1 roll put userdict /TBx2 3 -1 roll put
           userdict /TBy1 3 -1 roll put userdict /TBx1 3 -1 roll put
	   /Boxing true def } def
/ExtendTextBox { Boxing
    { gsave dup false charpath pathbbox
      dup TBy2 gt {userdict /TBy2 3 -1 roll put} {pop} ifelse
      dup TBx2 gt {userdict /TBx2 3 -1 roll put} {pop} ifelse
      dup TBy1 lt {userdict /TBy1 3 -1 roll put} {pop} ifelse
      dup TBx1 lt {userdict /TBx1 3 -1 roll put} {pop} ifelse
      grestore } if } def
/PopTextBox { newpath TBx1 TBxmargin sub TBy1 TBymargin sub M
               TBx1 TBxmargin sub TBy2 TBymargin add L
	       TBx2 TBxmargin add TBy2 TBymargin add L
	       TBx2 TBxmargin add TBy1 TBymargin sub L closepath } def
/DrawTextBox { PopTextBox stroke /Boxing false def} def
/FillTextBox { gsave PopTextBox 1 1 1 setrgbcolor fill grestore /Boxing false def} def
0 0 0 0 InitTextBox
/TBxmargin 20 def
/TBymargin 20 def
/Boxing false def
/textshow { ExtendTextBox Gshow } def
%
% redundant definitions for compatibility with prologue.ps older than 5.0.2
/LTB {BL [] LCb DL} def
/LTb {PL [] LCb DL} def
end
%%EndProlog
%%Page: 1 1
gnudict begin
gsave
doclip
0 0 translate
0.050 0.050 scale
0 setgray
newpath
BackgroundColor 0 lt 3 1 roll 0 lt exch 0 lt or or not {BackgroundColor C 1.000 0 0 7200.00 5040.00 BoxColFill} if
1.000 UL
LTb
LCb setrgbcolor
860 640 M
63 0 V
5916 0 R
-63 0 V
stroke
LTb
LCb setrgbcolor
860 1160 M
63 0 V
5916 0 R
-63 0 V
stroke
LTb
LCb setrgbcolor
860 1680 M
63 0 V
5916 0 R
-63 0 V
stroke
LTb
LCb setrgbcolor
860 2200 M
63 0 V
5916 0 R
-63 0 V
stroke
LTb
LCb setrgbcolor
860 2720 M
63 0 V
5916 0 R
-63 0 V
stroke
LTb
LCb setrgbcolor
860 3239 M
63 0 V
5916 0 R
-63 0 V
stroke
LTb
LCb setrgbcolor
860 3759 M
63 0 V
5916 0 R
-63 0 V
stroke
LTb
LCb setrgbcolor
860 4279 M
63 0 V
5916 0 R
-63 0 V
stroke
LTb
LCb setrgbcolor
860 4799 M
63 0 V
5916 0 R
-63 0 V
stroke
LTb
LCb setrgbcolor
860 640 M
0 63 V
0 4096 R
0 -63 V
stroke
LTb
LCb setrgbcolor
1714 640 M
0 63 V
0 4096 R
0 -63 V
stroke
LTb
LCb setrgbcolor
2568 640 M
0 63 V
0 4096 R
0 -63 V
stroke
LTb
LCb setrgbcolor
3422 640 M
0 63 V
0 4096 R
0 -63 V
stroke
LTb
LCb setrgbcolor
4277 640 M
0 63 V
0 4096 R
0 -63 V
stroke
LTb
LCb setrgbcolor
5131 640 M
0 63 V
0 4096 R
0 -63 V
stroke
LTb
LCb setrgbcolor
5985 640 M
0 63 V
0 4096 R
0 -63 V
stroke
LTb
LCb setrgbcolor
6839 640 M
0 63 V
0 4096 R
0 -63 V
stroke
LTb
LCb setrgbcolor
1.000 UL
LTb
LCb setrgbcolor
860 4799 N
860 640 L
5979 0 V
0 4159 V
-5979 0 V
Z stroke
1.000 UP
1.000 UL
LTb
LCb setrgbcolor
LCb setrgbcolor
LTb
LCb setrgbcolor
LTb
1.000 UP
1.000 UL
LTb
0.58 0.00 0.83 C LCb setrgbcolor
1.000 UP
1.000 UL
LTb
0.58 0.00 0.83 C 860 640 Pls
864 666 Pls
869 692 Pls
873 718 Pls
877 744 Pls
881 770 Pls
886 796 Pls
890 822 Pls
894 848 Pls
898 873 Pls
903 899 Pls
907 925 Pls
911 950 Pls
916 976 Pls
920 1002 Pls
924 1027 Pls
928 1052 Pls
933 1078 Pls
937 1103 Pls
941 1128 Pls
945 1153 Pls
950 1178 Pls
954 1203 Pls
958 1228 Pls
962 1252 Pls
967 1277 Pls
971 1301 Pls
975 1325 Pls
980 1350 Pls
984 1374 Pls
988 1398 Pls
992 1421 Pls
997 1445 Pls
1001 1469 Pls
1005 1492 Pls
1009 1515 Pls
1014 1538 Pls
1018 1561 Pls
1022 1584 Pls
1027 1607 Pls
1031 1629 Pls
1035 1651 Pls
1039 1674 Pls
1044 1696 Pls
1048 1717 Pls
1052 1739 Pls
1056 1761 Pls
1061 1782 Pls
1065 1803 Pls
1069 1824 Pls
1074 1845 Pls
1078 1866 Pls
1082 1886 Pls
1086 1907 Pls
1091 1927 Pls
1095 1947 Pls
1099 1967 Pls
1103 1987 Pls
1108 2006 Pls
1112 2026 Pls
1116 2045 Pls
1121 2064 Pls
1125 2083 Pls
1129 2101 Pls
1133 2120 Pls
1138 2138 Pls
1142 2156 Pls
1146 2174 Pls
1150 2192 Pls
1155 2210 Pls
1159 2228 Pls
1163 2245 Pls
1167 2262 Pls
1172 2279 Pls
1176 2296 Pls
1180 2313 Pls
1185 2329 Pls
1189 2346 Pls
1193 2362 Pls
1197 2378 Pls
1202 2394 Pls
1206 2410 Pls
1210 2425 Pls
1214 2441 Pls
1219 2456 Pls
1223 2471 Pls
1227 2486 Pls
1232 2501 Pls
1236 2516 Pls
1240 2530 Pls
1244 2545 Pls
1249 2559 Pls
1253 2573 Pls
1257 2587 Pls
1261 2601 Pls
1266 2615 Pls
1270 2629 Pls
1274 2642 Pls
1279 2655 Pls
1283 2668 Pls
1287 2682 Pls
1291 2694 Pls
1296 2707 Pls
1300 2720 Pls
1304 2733 Pls
1308 2745 Pls
1313 2757 Pls
1317 2769 Pls
1321 2781 Pls
1326 2793 Pls
1330 2805 Pls
1334 2817 Pls
1338 2829 Pls
1343 2840 Pls
1347 2851 Pls
1351 2863 Pls
1355 2874 Pls
1360 2885 Pls
1364 2896 Pls
1368 2906 Pls
1372 2917 Pls
1377 2928 Pls
1381 2938 Pls
1385 2949 Pls
1390 2959 Pls
1394 2969 Pls
1398 2979 Pls
1402 2989 Pls
1407 2999 Pls
1411 3009 Pls
1415 3019 Pls
1419 3028 Pls
1424 3038 Pls
1428 3047 Pls
1432 3057 Pls
1437 3066 Pls
1441 3075 Pls
1445 3084 Pls
1449 3093 Pls
1454 3102 Pls
1458 3111 Pls
1462 3120 Pls
1466 3128 Pls
1471 3137 Pls
1475 3145 Pls
1479 3154 Pls
1484 3162 Pls
1488 3170 Pls
1492 3179 Pls
1496 3187 Pls
1501 3195 Pls
1505 3203 Pls
1509 3210 Pls
1513 3218 Pls
1518 3226 Pls
1522 3234 Pls
1526 3241 Pls
1531 3249 Pls
1535 3256 Pls
1539 3264 Pls
1543 3271 Pls
1548 3278 Pls
1552 3286 Pls
1556 3293 Pls
1560 3300 Pls
1565 3307 Pls
1569 3314 Pls
1573 3321 Pls
1577 3327 Pls
1582 3334 Pls
1586 3341 Pls
1590 3348 Pls
1595 3354 Pls
1599 3361 Pls
1603 3367 Pls
1607 3374 Pls
1612 3380 Pls
1616 3386 Pls
1620 3393 Pls
1624 3399 Pls
1629 3405 Pls
1633 3411 Pls
1637 3417 Pls
1642 3423 Pls
1646 3429 Pls
1650 3435 Pls
1654 3441 Pls
1659 3447 Pls
1663 3452 Pls
1667 3458 Pls
1671 3464 Pls
1676 3469 Pls
1680 3475 Pls
1684 3480 Pls
1689 3486 Pls
1693 3491 Pls
1697 3497 Pls
1701 3502 Pls
1706 3507 Pls
1710 3513 Pls
1714 3518 Pls
1718 3523 Pls
1723 3528 Pls
1727 3533 Pls
1731 3538 Pls
1735 3543 Pls
1740 3548 Pls
1744 3553 Pls
1748 3558 Pls
1753 3563 Pls
1757 3568 Pls
1761 3573 Pls
1765 3577 Pls
1770 3582 Pls
1774 3587 Pls
1778 3591 Pls
1782 3596 Pls
1787 3601 Pls
1791 3605 Pls
1795 3610 Pls
1800 3614 Pls
1804 3619 Pls
1808 3623 Pls
1812 3627 Pls
1817 3632 Pls
1821 3636 Pls
1825 3640 Pls
1829 3644 Pls
1834 3649 Pls
1838 3653 Pls
1842 3657 Pls
1847 3661 Pls
1851 3665 Pls
1855 3669 Pls
1859 3673 Pls
1864 3677 Pls
1868 3681 Pls
1872 3685 Pls
1876 3689 Pls
1881 3693 Pls
1885 3697 Pls
1889 3701 Pls
1894 3705 Pls
1898 3708 Pls
1902 3712 Pls
1906 3716 Pls
1911 3719 Pls
1915 3723 Pls
1919 3727 Pls
1923 3730 Pls
1928 3734 Pls
1932 3738 Pls
1936 3741 Pls
1940 3745 Pls
1945 3748 Pls
1949 3752 Pls
1953 3755 Pls
1958 3759 Pls
1962 3762 Pls
1966 3765 Pls
1970 3769 Pls
1975 3772 Pls
1979 3775 Pls
1983 3779 Pls
1987 3782 Pls
1992 3785 Pls
1996 3788 Pls
2000 3792 Pls
2005 3795 Pls
2009 3798 Pls
2013 3801 Pls
2017 3804 Pls
2022 3807 Pls
2026 3810 Pls
2030 3813 Pls
2034 3817 Pls
2039 3820 Pls
2043 3823 Pls
2047 3826 Pls
2052 3828 Pls
2056 3831 Pls
2060 3834 Pls
2064 3837 Pls
2069 3840 Pls
2073 3843 Pls
2077 3846 Pls
2081 3849 Pls
2086 3852 Pls
2090 3854 Pls
2094 3857 Pls
2099 3860 Pls
2103 3863 Pls
2107 3865 Pls
2111 3868 Pls
2116 3871 Pls
2120 3874 Pls
2124 3876 Pls
2128 3879 Pls
2133 3882 Pls
2137 3884 Pls
2141 3887 Pls
2145 3889 Pls
2150 3892 Pls
2154 3894 Pls
2158 3897 Pls
2163 3900 Pls
2167 3902 Pls
2171 3905 Pls
2175 3907 Pls
2180 3910 Pls
2184 3912 Pls
2188 3914 Pls
2192 3917 Pls
2197 3919 Pls
2201 3922 Pls
2205 3924 Pls
2210 3926 Pls
2214 3929 Pls
2218 3931 Pls
2222 3933 Pls
2227 3936 Pls
2231 3938 Pls
2235 3940 Pls
2239 3943 Pls
2244 3945 Pls
2248 3947 Pls
2252 3949 Pls
2257 3952 Pls
2261 3954 Pls
2265 3956 Pls
2269 3958 Pls
2274 3960 Pls
2278 3963 Pls
2282 3965 Pls
2286 3967 Pls
2291 3969 Pls
2295 3971 Pls
2299 3973 Pls
2304 3975 Pls
2308 3977 Pls
2312 3980 Pls
2316 3982 Pls
2321 3984 Pls
2325 3986 Pls
2329 3988 Pls
2333 3990 Pls
2338 3992 Pls
2342 3994 Pls
2346 3996 Pls
2350 3998 Pls
2355 4000 Pls
2359 4002 Pls
2363 4004 Pls
2368 4006 Pls
2372 4007 Pls
2376 4009 Pls
2380 4011 Pls
2385 4013 Pls
2389 4015 Pls
2393 4017 Pls
2397 4019 Pls
2402 4021 Pls
2406 4023 Pls
2410 4024 Pls
2415 4026 Pls
2419 4028 Pls
2423 4030 Pls
2427 4032 Pls
2432 4033 Pls
2436 4035 Pls
2440 4037 Pls
2444 4039 Pls
2449 4040 Pls
2453 4042 Pls
2457 4044 Pls
2462 4046 Pls
2466 4047 Pls
2470 4049 Pls
2474 4051 Pls
2479 4053 Pls
2483 4054 Pls
2487 4056 Pls
2491 4058 Pls
2496 4059 Pls
2500 4061 Pls
2504 4063 Pls
2508 4064 Pls
2513 4066 Pls
2517 4067 Pls
2521 4069 Pls
2526 4071 Pls
2530 4072 Pls
2534 4074 Pls
2538 4075 Pls
2543 4077 Pls
2547 4079 Pls
2551 4080 Pls
2555 4082 Pls
2560 4083 Pls
2564 4085 Pls
2568 4086 Pls
2573 4088 Pls
2577 4089 Pls
2581 4091 Pls
2585 4092 Pls
2590 4094 Pls
2594 4095 Pls
2598 4097 Pls
2602 4098 Pls
2607 4100 Pls
2611 4101 Pls
2615 4103 Pls
2620 4104 Pls
2624 4106 Pls
2628 4107 Pls
2632 4108 Pls
2637 4110 Pls
2641 4111 Pls
2645 4113 Pls
2649 4114 Pls
2654 4116 Pls
2658 4117 Pls
2662 4118 Pls
2667 4120 Pls
2671 4121 Pls
2675 4122 Pls
2679 4124 Pls
2684 4125 Pls
2688 4126 Pls
2692 4128 Pls
2696 4129 Pls
2701 4130 Pls
2705 4132 Pls
2709 4133 Pls
2713 4134 Pls
2718 4136 Pls
2722 4137 Pls
2726 4138 Pls
2731 4140 Pls
2735 4141 Pls
2739 4142 Pls
2743 4143 Pls
2748 4145 Pls
2752 4146 Pls
2756 4147 Pls
2760 4149 Pls
2765 4150 Pls
2769 4151 Pls
2773 4152 Pls
2778 4153 Pls
2782 4155 Pls
2786 4156 Pls
2790 4157 Pls
2795 4158 Pls
2799 4160 Pls
2803 4161 Pls
2807 4162 Pls
2812 4163 Pls
2816 4164 Pls
2820 4165 Pls
2825 4167 Pls
2829 4168 Pls
2833 4169 Pls
2837 4170 Pls
2842 4171 Pls
2846 4172 Pls
2850 4174 Pls
2854 4175 Pls
2859 4176 Pls
2863 4177 Pls
2867 4178 Pls
2872 4179 Pls
2876 4180 Pls
2880 4182 Pls
2884 4183 Pls
2889 4184 Pls
2893 4185 Pls
2897 4186 Pls
2901 4187 Pls
2906 4188 Pls
2910 4189 Pls
2914 4190 Pls
2918 4191 Pls
2923 4192 Pls
2927 4193 Pls
2931 4195 Pls
2936 4196 Pls
2940 4197 Pls
2944 4198 Pls
2948 4199 Pls
2953 4200 Pls
2957 4201 Pls
2961 4202 Pls
2965 4203 Pls
2970 4204 Pls
2974 4205 Pls
2978 4206 Pls
2983 4207 Pls
2987 4208 Pls
2991 4209 Pls
2995 4210 Pls
3000 4211 Pls
3004 4212 Pls
3008 4213 Pls
3012 4214 Pls
3017 4215 Pls
3021 4216 Pls
3025 4217 Pls
3030 4218 Pls
3034 4219 Pls
3038 4220 Pls
3042 4221 Pls
3047 4222 Pls
3051 4223 Pls
3055 4224 Pls
3059 4225 Pls
3064 4226 Pls
3068 4226 Pls
3072 4227 Pls
3077 4228 Pls
3081 4229 Pls
3085 4230 Pls
3089 4231 Pls
3094 4232 Pls
3098 4233 Pls
3102 4234 Pls
3106 4235 Pls
3111 4236 Pls
3115 4237 Pls
3119 4237 Pls
3123 4238 Pls
3128 4239 Pls
3132 4240 Pls
3136 4241 Pls
3141 4242 Pls
3145 4243 Pls
3149 4244 Pls
3153 4245 Pls
3158 4245 Pls
3162 4246 Pls
3166 4247 Pls
3170 4248 Pls
3175 4249 Pls
3179 4250 Pls
3183 4251 Pls
3188 4251 Pls
3192 4252 Pls
3196 4253 Pls
3200 4254 Pls
3205 4255 Pls
3209 4256 Pls
3213 4256 Pls
3217 4257 Pls
3222 4258 Pls
3226 4259 Pls
3230 4260 Pls
3235 4261 Pls
3239 4261 Pls
3243 4262 Pls
3247 4263 Pls
3252 4264 Pls
3256 4265 Pls
3260 4265 Pls
3264 4266 Pls
3269 4267 Pls
3273 4268 Pls
3277 4269 Pls
3281 4269 Pls
3286 4270 Pls
3290 4271 Pls
3294 4272 Pls
3299 4272 Pls
3303 4273 Pls
3307 4274 Pls
3311 4275 Pls
3316 4276 Pls
3320 4276 Pls
3324 4277 Pls
3328 4278 Pls
3333 4279 Pls
3337 4279 Pls
3341 4280 Pls
3346 4281 Pls
3350 4282 Pls
3354 4282 Pls
3358 4283 Pls
3363 4284 Pls
3367 4284 Pls
3371 4285 Pls
3375 4286 Pls
3380 4287 Pls
3384 4287 Pls
3388 4288 Pls
3393 4289 Pls
3397 4290 Pls
3401 4290 Pls
3405 4291 Pls
3410 4292 Pls
3414 4292 Pls
3418 4293 Pls
3422 4294 Pls
3427 4295 Pls
3431 4295 Pls
3435 4296 Pls
3440 4297 Pls
3444 4297 Pls
3448 4298 Pls
3452 4299 Pls
3457 4299 Pls
3461 4300 Pls
3465 4301 Pls
3469 4301 Pls
3474 4302 Pls
3478 4303 Pls
3482 4303 Pls
3486 4304 Pls
3491 4305 Pls
3495 4305 Pls
3499 4306 Pls
3504 4307 Pls
3508 4307 Pls
3512 4308 Pls
3516 4309 Pls
3521 4309 Pls
3525 4310 Pls
3529 4311 Pls
3533 4311 Pls
3538 4312 Pls
3542 4313 Pls
3546 4313 Pls
3551 4314 Pls
3555 4315 Pls
3559 4315 Pls
3563 4316 Pls
3568 4316 Pls
3572 4317 Pls
3576 4318 Pls
3580 4318 Pls
3585 4319 Pls
3589 4320 Pls
3593 4320 Pls
3598 4321 Pls
3602 4321 Pls
3606 4322 Pls
3610 4323 Pls
3615 4323 Pls
3619 4324 Pls
3623 4324 Pls
3627 4325 Pls
3632 4326 Pls
3636 4326 Pls
3640 4327 Pls
3645 4327 Pls
3649 4328 Pls
3653 4329 Pls
3657 4329 Pls
3662 4330 Pls
3666 4330 Pls
3670 4331 Pls
3674 4332 Pls
3679 4332 Pls
3683 4333 Pls
3687 4333 Pls
3691 4334 Pls
3696 4335 Pls
3700 4335 Pls
3704 4336 Pls
3709 4336 Pls
3713 4337 Pls
3717 4337 Pls
3721 4338 Pls
3726 4339 Pls
3730 4339 Pls
3734 4340 Pls
3738 4340 Pls
3743 4341 Pls
3747 4341 Pls
3751 4342 Pls
3756 4342 Pls
3760 4343 Pls
3764 4344 Pls
3768 4344 Pls
3773 4345 Pls
3777 4345 Pls
3781 4346 Pls
3785 4346 Pls
3790 4347 Pls
3794 4347 Pls
3798 4348 Pls
3803 4348 Pls
3807 4349 Pls
3811 4350 Pls
3815 4350 Pls
3820 4351 Pls
3824 4351 Pls
3828 4352 Pls
3832 4352 Pls
3837 4353 Pls
3841 4353 Pls
3845 4354 Pls
3850 4354 Pls
3854 4355 Pls
3858 4355 Pls
3862 4356 Pls
3867 4356 Pls
3871 4357 Pls
3875 4357 Pls
3879 4358 Pls
3884 4358 Pls
3888 4359 Pls
3892 4359 Pls
3896 4360 Pls
3901 4360 Pls
3905 4361 Pls
3909 4361 Pls
3914 4362 Pls
3918 4362 Pls
3922 4363 Pls
3926 4363 Pls
3931 4364 Pls
3935 4364 Pls
3939 4365 Pls
3943 4365 Pls
3948 4366 Pls
3952 4366 Pls
3956 4367 Pls
3961 4367 Pls
3965 4368 Pls
3969 4368 Pls
3973 4369 Pls
3978 4369 Pls
3982 4370 Pls
3986 4370 Pls
3990 4371 Pls
3995 4371 Pls
3999 4372 Pls
4003 4372 Pls
4008 4373 Pls
4012 4373 Pls
4016 4373 Pls
4020 4374 Pls
4025 4374 Pls
4029 4375 Pls
4033 4375 Pls
4037 4376 Pls
4042 4376 Pls
4046 4377 Pls
4050 4377 Pls
4054 4378 Pls
4059 4378 Pls
4063 4379 Pls
4067 4379 Pls
4072 4379 Pls
4076 4380 Pls
4080 4380 Pls
4084 4381 Pls
4089 4381 Pls
4093 4382 Pls
4097 4382 Pls
4101 4383 Pls
4106 4383 Pls
4110 4383 Pls
4114 4384 Pls
4119 4384 Pls
4123 4385 Pls
4127 4385 Pls
4131 4386 Pls
4136 4386 Pls
4140 4387 Pls
4144 4387 Pls
4148 4387 Pls
4153 4388 Pls
4157 4388 Pls
4161 4389 Pls
4166 4389 Pls
4170 4390 Pls
4174 4390 Pls
4178 4390 Pls
4183 4391 Pls
4187 4391 Pls
4191 4392 Pls
4195 4392 Pls
4200 4392 Pls
4204 4393 Pls
4208 4393 Pls
4213 4394 Pls
4217 4394 Pls
4221 4395 Pls
4225 4395 Pls
4230 4395 Pls
4234 4396 Pls
4238 4396 Pls
4242 4397 Pls
4247 4397 Pls
4251 4397 Pls
4255 4398 Pls
4259 4398 Pls
4264 4399 Pls
4268 4399 Pls
4272 4399 Pls
4277 4400 Pls
4281 4400 Pls
4285 4401 Pls
4289 4401 Pls
4294 4401 Pls
4298 4402 Pls
4302 4402 Pls
4306 4403 Pls
4311 4403 Pls
4315 4403 Pls
4319 4404 Pls
4324 4404 Pls
4328 4405 Pls
4332 4405 Pls
4336 4405 Pls
4341 4406 Pls
4345 4406 Pls
4349 4407 Pls
4353 4407 Pls
4358 4407 Pls
4362 4408 Pls
4366 4408 Pls
4371 4408 Pls
4375 4409 Pls
4379 4409 Pls
4383 4410 Pls
4388 4410 Pls
4392 4410 Pls
4396 4411 Pls
4400 4411 Pls
4405 4411 Pls
4409 4412 Pls
4413 4412 Pls
4418 4413 Pls
4422 4413 Pls
4426 4413 Pls
4430 4414 Pls
4435 4414 Pls
4439 4414 Pls
4443 4415 Pls
4447 4415 Pls
4452 4415 Pls
4456 4416 Pls
4460 4416 Pls
4464 4417 Pls
4469 4417 Pls
4473 4417 Pls
4477 4418 Pls
4482 4418 Pls
4486 4418 Pls
4490 4419 Pls
4494 4419 Pls
4499 4419 Pls
4503 4420 Pls
4507 4420 Pls
4511 4420 Pls
4516 4421 Pls
4520 4421 Pls
4524 4421 Pls
4529 4422 Pls
4533 4422 Pls
4537 4423 Pls
4541 4423 Pls
4546 4423 Pls
4550 4424 Pls
4554 4424 Pls
4558 4424 Pls
4563 4425 Pls
4567 4425 Pls
4571 4425 Pls
4576 4426 Pls
4580 4426 Pls
4584 4426 Pls
4588 4427 Pls
4593 4427 Pls
4597 4427 Pls
4601 4428 Pls
4605 4428 Pls
4610 4428 Pls
4614 4429 Pls
4618 4429 Pls
4622 4429 Pls
4627 4430 Pls
4631 4430 Pls
4635 4430 Pls
4640 4431 Pls
4644 4431 Pls
4648 4431 Pls
4652 4432 Pls
4657 4432 Pls
4661 4432 Pls
4665 4433 Pls
4669 4433 Pls
4674 4433 Pls
4678 4434 Pls
4682 4434 Pls
4687 4434 Pls
4691 4435 Pls
4695 4435 Pls
4699 4435 Pls
4704 4435 Pls
4708 4436 Pls
4712 4436 Pls
4716 4436 Pls
4721 4437 Pls
4725 4437 Pls
4729 4437 Pls
4734 4438 Pls
4738 4438 Pls
4742 4438 Pls
4746 4439 Pls
4751 4439 Pls
4755 4439 Pls
4759 4440 Pls
4763 4440 Pls
4768 4440 Pls
4772 4440 Pls
4776 4441 Pls
4781 4441 Pls
4785 4441 Pls
4789 4442 Pls
4793 4442 Pls
4798 4442 Pls
4802 4443 Pls
4806 4443 Pls
4810 4443 Pls
4815 4443 Pls
4819 4444 Pls
4823 4444 Pls
4827 4444 Pls
4832 4445 Pls
4836 4445 Pls
4840 4445 Pls
4845 4446 Pls
4849 4446 Pls
4853 4446 Pls
4857 4446 Pls
4862 4447 Pls
4866 4447 Pls
4870 4447 Pls
4874 4448 Pls
4879 4448 Pls
4883 4448 Pls
4887 4448 Pls
4892 4449 Pls
4896 4449 Pls
4900 4449 Pls
4904 4450 Pls
4909 4450 Pls
4913 4450 Pls
4917 4450 Pls
4921 4451 Pls
4926 4451 Pls
4930 4451 Pls
4934 4452 Pls
4939 4452 Pls
4943 4452 Pls
4947 4452 Pls
4951 4453 Pls
4956 4453 Pls
4960 4453 Pls
4964 4454 Pls
4968 4454 Pls
4973 4454 Pls
4977 4454 Pls
4981 4455 Pls
4986 4455 Pls
4990 4455 Pls
4994 4456 Pls
4998 4456 Pls
5003 4456 Pls
5007 4456 Pls
5011 4457 Pls
5015 4457 Pls
5020 4457 Pls
5024 4457 Pls
5028 4458 Pls
5032 4458 Pls
5037 4458 Pls
5041 4458 Pls
5045 4459 Pls
5050 4459 Pls
5054 4459 Pls
5058 4460 Pls
5062 4460 Pls
5067 4460 Pls
5071 4460 Pls
5075 4461 Pls
5079 4461 Pls
5084 4461 Pls
5088 4461 Pls
5092 4462 Pls
5097 4462 Pls
5101 4462 Pls
5105 4462 Pls
5109 4463 Pls
5114 4463 Pls
5118 4463 Pls
5122 4464 Pls
5126 4464 Pls
5131 4464 Pls
5135 4464 Pls
5139 4465 Pls
5144 4465 Pls
5148 4465 Pls
5152 4465 Pls
5156 4466 Pls
5161 4466 Pls
5165 4466 Pls
5169 4466 Pls
5173 4467 Pls
5178 4467 Pls
5182 4467 Pls
5186 4467 Pls
5191 4468 Pls
5195 4468 Pls
5199 4468 Pls
5203 4468 Pls
5208 4469 Pls
5212 4469 Pls
5216 4469 Pls
5220 4469 Pls
5225 4470 Pls
5229 4470 Pls
5233 4470 Pls
5237 4470 Pls
5242 4471 Pls
5246 4471 Pls
5250 4471 Pls
5255 4471 Pls
5259 4472 Pls
5263 4472 Pls
5267 4472 Pls
5272 4472 Pls
5276 4472 Pls
5280 4473 Pls
5284 4473 Pls
5289 4473 Pls
5293 4473 Pls
5297 4474 Pls
5302 4474 Pls
5306 4474 Pls
5310 4474 Pls
5314 4475 Pls
5319 4475 Pls
5323 4475 Pls
5327 4475 Pls
5331 4476 Pls
5336 4476 Pls
5340 4476 Pls
5344 4476 Pls
5349 4477 Pls
5353 4477 Pls
5357 4477 Pls
5361 4477 Pls
5366 4477 Pls
5370 4478 Pls
5374 4478 Pls
5378 4478 Pls
5383 4478 Pls
5387 4479 Pls
5391 4479 Pls
5395 4479 Pls
5400 4479 Pls
5404 4480 Pls
5408 4480 Pls
5413 4480 Pls
5417 4480 Pls
5421 4480 Pls
5425 4481 Pls
5430 4481 Pls
5434 4481 Pls
5438 4481 Pls
5442 4482 Pls
5447 4482 Pls
5451 4482 Pls
5455 4482 Pls
5460 4482 Pls
5464 4483 Pls
5468 4483 Pls
5472 4483 Pls
5477 4483 Pls
5481 4484 Pls
5485 4484 Pls
5489 4484 Pls
5494 4484 Pls
5498 4484 Pls
5502 4485 Pls
5507 4485 Pls
5511 4485 Pls
5515 4485 Pls
5519 4486 Pls
5524 4486 Pls
5528 4486 Pls
5532 4486 Pls
5536 4486 Pls
5541 4487 Pls
5545 4487 Pls
5549 4487 Pls
5554 4487 Pls
5558 4487 Pls
5562 4488 Pls
5566 4488 Pls
5571 4488 Pls
5575 4488 Pls
5579 4488 Pls
5583 4489 Pls
5588 4489 Pls
5592 4489 Pls
5596 4489 Pls
5600 4490 Pls
5605 4490 Pls
5609 4490 Pls
5613 4490 Pls
5618 4490 Pls
5622 4491 Pls
5626 4491 Pls
5630 4491 Pls
5635 4491 Pls
5639 4491 Pls
5643 4492 Pls
5647 4492 Pls
5652 4492 Pls
5656 4492 Pls
5660 4492 Pls
5665 4493 Pls
5669 4493 Pls
5673 4493 Pls
5677 4493 Pls
5682 4493 Pls
5686 4494 Pls
5690 4494 Pls
5694 4494 Pls
5699 4494 Pls
5703 4494 Pls
5707 4495 Pls
5712 4495 Pls
5716 4495 Pls
5720 4495 Pls
5724 4495 Pls
5729 4496 Pls
5733 4496 Pls
5737 4496 Pls
5741 4496 Pls
5746 4496 Pls
5750 4497 Pls
5754 4497 Pls
5759 4497 Pls
5763 4497 Pls
5767 4497 Pls
5771 4498 Pls
5776 4498 Pls
5780 4498 Pls
5784 4498 Pls
5788 4498 Pls
5793 4499 Pls
5797 4499 Pls
5801 4499 Pls
5805 4499 Pls
5810 4499 Pls
5814 4500 Pls
5818 4500 Pls
5823 4500 Pls
5827 4500 Pls
5831 4500 Pls
5835 4501 Pls
5840 4501 Pls
5844 4501 Pls
5848 4501 Pls
5852 4501 Pls
5857 4501 Pls
5861 4502 Pls
5865 4502 Pls
5870 4502 Pls
5874 4502 Pls
5878 4502 Pls
5882 4503 Pls
5887 4503 Pls
5891 4503 Pls
5895 4503 Pls
5899 4503 Pls
5904 4504 Pls
5908 4504 Pls
5912 4504 Pls
5917 4504 Pls
5921 4504 Pls
5925 4504 Pls
5929 4505 Pls
5934 4505 Pls
5938 4505 Pls
5942 4505 Pls
5946 4505 Pls
5951 4506 Pls
5955 4506 Pls
5959 4506 Pls
5964 4506 Pls
5968 4506 Pls
5972 4506 Pls
5976 4507 Pls
5981 4507 Pls
5985 4507 Pls
5989 4507 Pls
5993 4507 Pls
5998 4508 Pls
6002 4508 Pls
6006 4508 Pls
6010 4508 Pls
6015 4508 Pls
6019 4508 Pls
6023 4509 Pls
6028 4509 Pls
6032 4509 Pls
6036 4509 Pls
6040 4509 Pls
6045 4509 Pls
6049 4510 Pls
6053 4510 Pls
6057 4510 Pls
6062 4510 Pls
6066 4510 Pls
6070 4511 Pls
6075 4511 Pls
6079 4511 Pls
6083 4511 Pls
6087 4511 Pls
6092 4511 Pls
6096 4512 Pls
6100 4512 Pls
6104 4512 Pls
6109 4512 Pls
6113 4512 Pls
6117 4512 Pls
6122 4513 Pls
6126 4513 Pls
6130 4513 Pls
6134 4513 Pls
6139 4513 Pls
6143 4513 Pls
6147 4514 Pls
6151 4514 Pls
6156 4514 Pls
6160 4514 Pls
6164 4514 Pls
6168 4514 Pls
6173 4515 Pls
6177 4515 Pls
6181 4515 Pls
6186 4515 Pls
6190 4515 Pls
6194 4515 Pls
6198 4516 Pls
6203 4516 Pls
6207 4516 Pls
6211 4516 Pls
6215 4516 Pls
6220 4516 Pls
6224 4517 Pls
6327 1003 Pls
1.000 UL
LTb
0.00 0.62 0.45 C LCb setrgbcolor
1.000 UL
LTb
0.00 0.62 0.45 C 6056 803 M
543 0 V
860 640 M
4 26 V
5 26 V
4 26 V
4 26 V
4 26 V
5 26 V
4 26 V
4 26 V
4 25 V
5 26 V
4 26 V
4 25 V
5 26 V
4 26 V
4 25 V
4 25 V
5 26 V
4 25 V
4 25 V
4 25 V
5 25 V
4 25 V
4 25 V
4 24 V
5 25 V
4 24 V
4 24 V
5 25 V
4 24 V
4 24 V
4 23 V
5 24 V
4 24 V
4 23 V
4 23 V
5 23 V
4 23 V
4 23 V
5 23 V
4 22 V
4 22 V
4 23 V
5 22 V
4 21 V
4 22 V
4 22 V
5 21 V
4 21 V
4 21 V
5 21 V
4 21 V
4 20 V
4 21 V
5 20 V
4 20 V
4 20 V
4 20 V
5 19 V
4 20 V
4 19 V
5 19 V
4 19 V
4 18 V
4 19 V
5 18 V
4 18 V
4 18 V
4 18 V
5 18 V
4 18 V
4 17 V
4 17 V
5 17 V
4 17 V
4 17 V
5 16 V
4 17 V
4 16 V
4 16 V
5 16 V
4 16 V
4 15 V
4 16 V
5 15 V
4 15 V
4 15 V
5 15 V
4 15 V
4 14 V
4 15 V
5 14 V
4 14 V
4 14 V
4 14 V
5 14 V
4 14 V
4 13 V
5 13 V
4 13 V
4 14 V
4 12 V
5 13 V
stroke 1296 2707 M
4 13 V
4 13 V
4 12 V
5 12 V
4 12 V
4 12 V
5 12 V
4 12 V
4 12 V
4 12 V
5 11 V
4 11 V
4 12 V
4 11 V
5 11 V
4 11 V
4 10 V
4 11 V
5 11 V
4 10 V
4 11 V
5 10 V
4 10 V
4 10 V
4 10 V
5 10 V
4 10 V
4 10 V
4 9 V
5 10 V
4 9 V
4 10 V
5 9 V
4 9 V
4 9 V
4 9 V
5 9 V
4 9 V
4 9 V
4 8 V
5 9 V
4 8 V
4 9 V
5 8 V
4 8 V
4 9 V
4 8 V
5 8 V
4 8 V
4 7 V
4 8 V
5 8 V
4 8 V
4 7 V
5 8 V
4 7 V
4 8 V
4 7 V
5 7 V
4 8 V
4 7 V
4 7 V
5 7 V
4 7 V
4 7 V
4 6 V
5 7 V
4 7 V
4 7 V
5 6 V
4 7 V
4 6 V
4 7 V
5 6 V
4 6 V
4 7 V
4 6 V
5 6 V
4 6 V
4 6 V
5 6 V
4 6 V
4 6 V
4 6 V
5 6 V
4 5 V
4 6 V
4 6 V
5 5 V
4 6 V
4 5 V
5 6 V
4 5 V
4 6 V
4 5 V
5 5 V
4 6 V
4 5 V
4 5 V
5 5 V
4 5 V
4 5 V
4 5 V
5 5 V
stroke 1740 3548 M
4 5 V
4 5 V
5 5 V
4 5 V
4 5 V
4 4 V
5 5 V
4 5 V
4 4 V
4 5 V
5 5 V
4 4 V
4 5 V
5 4 V
4 5 V
4 4 V
4 4 V
5 5 V
4 4 V
4 4 V
4 4 V
5 5 V
4 4 V
4 4 V
5 4 V
4 4 V
4 4 V
4 4 V
5 4 V
4 4 V
4 4 V
4 4 V
5 4 V
4 4 V
4 4 V
5 4 V
4 3 V
4 4 V
4 4 V
5 3 V
4 4 V
4 4 V
4 3 V
5 4 V
4 4 V
4 3 V
4 4 V
5 3 V
4 4 V
4 3 V
5 4 V
4 3 V
4 3 V
4 4 V
5 3 V
4 3 V
4 4 V
4 3 V
5 3 V
4 3 V
4 4 V
5 3 V
4 3 V
4 3 V
4 3 V
5 3 V
4 3 V
4 3 V
4 4 V
5 3 V
4 3 V
4 3 V
5 2 V
4 3 V
4 3 V
4 3 V
5 3 V
4 3 V
4 3 V
4 3 V
5 3 V
4 2 V
4 3 V
5 3 V
4 3 V
4 2 V
4 3 V
5 3 V
4 3 V
4 2 V
4 3 V
5 3 V
4 2 V
4 3 V
4 2 V
5 3 V
4 2 V
4 3 V
5 3 V
4 2 V
4 3 V
4 2 V
5 3 V
4 2 V
stroke 2184 3912 M
4 2 V
4 3 V
5 2 V
4 3 V
4 2 V
5 2 V
4 3 V
4 2 V
4 2 V
5 3 V
4 2 V
4 2 V
4 3 V
5 2 V
4 2 V
4 2 V
5 3 V
4 2 V
4 2 V
4 2 V
5 2 V
4 3 V
4 2 V
4 2 V
5 2 V
4 2 V
4 2 V
5 2 V
4 2 V
4 3 V
4 2 V
5 2 V
4 2 V
4 2 V
4 2 V
5 2 V
4 2 V
4 2 V
4 2 V
5 2 V
4 2 V
4 2 V
5 2 V
4 1 V
4 2 V
4 2 V
5 2 V
4 2 V
4 2 V
4 2 V
5 2 V
4 2 V
4 1 V
5 2 V
4 2 V
4 2 V
4 2 V
5 1 V
4 2 V
4 2 V
4 2 V
5 1 V
4 2 V
4 2 V
5 2 V
4 1 V
4 2 V
4 2 V
5 2 V
4 1 V
4 2 V
4 2 V
5 1 V
4 2 V
4 2 V
4 1 V
5 2 V
4 1 V
4 2 V
5 2 V
4 1 V
4 2 V
4 1 V
5 2 V
4 2 V
4 1 V
4 2 V
5 1 V
4 2 V
4 1 V
5 2 V
4 1 V
4 2 V
4 1 V
5 2 V
4 1 V
4 2 V
4 1 V
5 2 V
4 1 V
4 2 V
5 1 V
4 2 V
4 1 V
stroke 2628 4107 M
4 1 V
5 2 V
4 1 V
4 2 V
4 1 V
5 2 V
4 1 V
4 1 V
5 2 V
4 1 V
4 1 V
4 2 V
5 1 V
4 1 V
4 2 V
4 1 V
5 1 V
4 2 V
4 1 V
4 1 V
5 2 V
4 1 V
4 1 V
5 2 V
4 1 V
4 1 V
4 1 V
5 2 V
4 1 V
4 1 V
4 2 V
5 1 V
4 1 V
4 1 V
5 1 V
4 2 V
4 1 V
4 1 V
5 1 V
4 2 V
4 1 V
4 1 V
5 1 V
4 1 V
4 1 V
5 2 V
4 1 V
4 1 V
4 1 V
5 1 V
4 1 V
4 2 V
4 1 V
5 1 V
4 1 V
4 1 V
5 1 V
4 1 V
4 2 V
4 1 V
5 1 V
4 1 V
4 1 V
4 1 V
5 1 V
4 1 V
4 1 V
4 1 V
5 1 V
4 1 V
4 2 V
5 1 V
4 1 V
4 1 V
4 1 V
5 1 V
4 1 V
4 1 V
4 1 V
5 1 V
4 1 V
4 1 V
5 1 V
4 1 V
4 1 V
4 1 V
5 1 V
4 1 V
4 1 V
4 1 V
5 1 V
4 1 V
4 1 V
5 1 V
4 1 V
4 1 V
4 1 V
5 1 V
4 1 V
4 1 V
4 1 V
5 1 V
4 0 V
4 1 V
stroke 3072 4227 M
5 1 V
4 1 V
4 1 V
4 1 V
5 1 V
4 1 V
4 1 V
4 1 V
5 1 V
4 1 V
4 0 V
4 1 V
5 1 V
4 1 V
4 1 V
5 1 V
4 1 V
4 1 V
4 1 V
5 0 V
4 1 V
4 1 V
4 1 V
5 1 V
4 1 V
4 1 V
5 0 V
4 1 V
4 1 V
4 1 V
5 1 V
4 1 V
4 0 V
4 1 V
5 1 V
4 1 V
4 1 V
5 1 V
4 0 V
4 1 V
4 1 V
5 1 V
4 1 V
4 0 V
4 1 V
5 1 V
4 1 V
4 1 V
4 0 V
5 1 V
4 1 V
4 1 V
5 0 V
4 1 V
4 1 V
4 1 V
5 1 V
4 0 V
4 1 V
4 1 V
5 1 V
4 0 V
4 1 V
5 1 V
4 1 V
4 0 V
4 1 V
5 1 V
4 0 V
4 1 V
4 1 V
5 1 V
4 0 V
4 1 V
5 1 V
4 1 V
4 0 V
4 1 V
5 1 V
4 0 V
4 1 V
4 1 V
5 1 V
4 0 V
4 1 V
5 1 V
4 0 V
4 1 V
4 1 V
5 0 V
4 1 V
4 1 V
4 0 V
5 1 V
4 1 V
4 0 V
4 1 V
5 1 V
4 0 V
4 1 V
5 1 V
4 0 V
4 1 V
4 1 V
stroke 3516 4309 M
5 0 V
4 1 V
4 1 V
4 0 V
5 1 V
4 1 V
4 0 V
5 1 V
4 1 V
4 0 V
4 1 V
5 0 V
4 1 V
4 1 V
4 0 V
5 1 V
4 1 V
4 0 V
5 1 V
4 0 V
4 1 V
4 1 V
5 0 V
4 1 V
4 0 V
4 1 V
5 1 V
4 0 V
4 1 V
5 0 V
4 1 V
4 1 V
4 0 V
5 1 V
4 0 V
4 1 V
4 1 V
5 0 V
4 1 V
4 0 V
4 1 V
5 1 V
4 0 V
4 1 V
5 0 V
4 1 V
4 0 V
4 1 V
5 1 V
4 0 V
4 1 V
4 0 V
5 1 V
4 0 V
4 1 V
5 0 V
4 1 V
4 1 V
4 0 V
5 1 V
4 0 V
4 1 V
4 0 V
5 1 V
4 0 V
4 1 V
5 0 V
4 1 V
4 1 V
4 0 V
5 1 V
4 0 V
4 1 V
4 0 V
5 1 V
4 0 V
4 1 V
5 0 V
4 1 V
4 0 V
4 1 V
5 0 V
4 1 V
4 0 V
4 1 V
5 0 V
4 1 V
4 0 V
4 1 V
5 0 V
4 1 V
4 0 V
5 1 V
4 0 V
4 1 V
4 0 V
5 1 V
4 0 V
4 1 V
4 0 V
5 1 V
4 0 V
4 1 V
5 0 V
stroke 3961 4367 M
4 1 V
4 0 V
4 1 V
5 0 V
4 1 V
4 0 V
4 1 V
5 0 V
4 1 V
4 0 V
5 1 V
4 0 V
4 0 V
4 1 V
5 0 V
4 1 V
4 0 V
4 1 V
5 0 V
4 1 V
4 0 V
4 1 V
5 0 V
4 1 V
4 0 V
5 0 V
4 1 V
4 0 V
4 1 V
5 0 V
4 1 V
4 0 V
4 1 V
5 0 V
4 0 V
4 1 V
5 0 V
4 1 V
4 0 V
4 1 V
5 0 V
4 1 V
4 0 V
4 0 V
5 1 V
4 0 V
4 1 V
5 0 V
4 1 V
4 0 V
4 0 V
5 1 V
4 0 V
4 1 V
4 0 V
5 0 V
4 1 V
4 0 V
5 1 V
4 0 V
4 1 V
4 0 V
5 0 V
4 1 V
4 0 V
4 1 V
5 0 V
4 0 V
4 1 V
4 0 V
5 1 V
4 0 V
4 0 V
5 1 V
4 0 V
4 1 V
4 0 V
5 0 V
4 1 V
4 0 V
4 1 V
5 0 V
4 0 V
4 1 V
5 0 V
4 1 V
4 0 V
4 0 V
5 1 V
4 0 V
4 1 V
4 0 V
5 0 V
4 1 V
4 0 V
5 0 V
4 1 V
4 0 V
4 1 V
5 0 V
4 0 V
4 1 V
4 0 V
5 0 V
stroke 4405 4411 M
4 1 V
4 0 V
5 1 V
4 0 V
4 0 V
4 1 V
5 0 V
4 0 V
4 1 V
4 0 V
5 0 V
4 1 V
4 0 V
4 1 V
5 0 V
4 0 V
4 1 V
5 0 V
4 0 V
4 1 V
4 0 V
5 0 V
4 1 V
4 0 V
4 0 V
5 1 V
4 0 V
4 0 V
5 1 V
4 0 V
4 1 V
4 0 V
5 0 V
4 1 V
4 0 V
4 0 V
5 1 V
4 0 V
4 0 V
5 1 V
4 0 V
4 0 V
4 1 V
5 0 V
4 0 V
4 1 V
4 0 V
5 0 V
4 1 V
4 0 V
4 0 V
5 1 V
4 0 V
4 0 V
5 1 V
4 0 V
4 0 V
4 1 V
5 0 V
4 0 V
4 1 V
4 0 V
5 0 V
4 1 V
4 0 V
5 0 V
4 0 V
4 1 V
4 0 V
5 0 V
4 1 V
4 0 V
4 0 V
5 1 V
4 0 V
4 0 V
5 1 V
4 0 V
4 0 V
4 1 V
5 0 V
4 0 V
4 1 V
4 0 V
5 0 V
4 0 V
4 1 V
5 0 V
4 0 V
4 1 V
4 0 V
5 0 V
4 1 V
4 0 V
4 0 V
5 0 V
4 1 V
4 0 V
4 0 V
5 1 V
4 0 V
4 0 V
5 1 V
4 0 V
stroke 4849 4446 M
4 0 V
4 0 V
5 1 V
4 0 V
4 0 V
4 1 V
5 0 V
4 0 V
4 0 V
5 1 V
4 0 V
4 0 V
4 1 V
5 0 V
4 0 V
4 0 V
4 1 V
5 0 V
4 0 V
4 1 V
5 0 V
4 0 V
4 0 V
4 1 V
5 0 V
4 0 V
4 1 V
4 0 V
5 0 V
4 0 V
4 1 V
5 0 V
4 0 V
4 1 V
4 0 V
5 0 V
4 0 V
4 1 V
4 0 V
5 0 V
4 0 V
4 1 V
4 0 V
5 0 V
4 0 V
4 1 V
5 0 V
4 0 V
4 1 V
4 0 V
5 0 V
4 0 V
4 1 V
4 0 V
5 0 V
4 0 V
4 1 V
5 0 V
4 0 V
4 0 V
4 1 V
5 0 V
4 0 V
4 0 V
4 1 V
5 0 V
4 0 V
4 1 V
5 0 V
4 0 V
4 0 V
4 1 V
5 0 V
4 0 V
4 0 V
4 1 V
5 0 V
4 0 V
4 0 V
5 1 V
4 0 V
4 0 V
4 0 V
5 1 V
4 0 V
4 0 V
4 0 V
5 1 V
4 0 V
4 0 V
4 0 V
5 1 V
4 0 V
4 0 V
5 0 V
4 1 V
4 0 V
4 0 V
5 0 V
4 0 V
4 1 V
4 0 V
5 0 V
4 0 V
stroke 5293 4473 M
4 1 V
5 0 V
4 0 V
4 0 V
4 1 V
5 0 V
4 0 V
4 0 V
4 1 V
5 0 V
4 0 V
4 0 V
5 1 V
4 0 V
4 0 V
4 0 V
5 0 V
4 1 V
4 0 V
4 0 V
5 0 V
4 1 V
4 0 V
4 0 V
5 0 V
4 1 V
4 0 V
5 0 V
4 0 V
4 0 V
4 1 V
5 0 V
4 0 V
4 0 V
4 1 V
5 0 V
4 0 V
4 0 V
5 0 V
4 1 V
4 0 V
4 0 V
5 0 V
4 1 V
4 0 V
4 0 V
5 0 V
4 0 V
4 1 V
5 0 V
4 0 V
4 0 V
4 0 V
5 1 V
4 0 V
4 0 V
4 0 V
5 1 V
4 0 V
4 0 V
5 0 V
4 0 V
4 1 V
4 0 V
5 0 V
4 0 V
4 0 V
4 1 V
5 0 V
4 0 V
4 0 V
4 1 V
5 0 V
4 0 V
4 0 V
5 0 V
4 1 V
4 0 V
4 0 V
5 0 V
4 0 V
4 1 V
4 0 V
5 0 V
4 0 V
4 0 V
5 1 V
4 0 V
4 0 V
4 0 V
5 0 V
4 1 V
4 0 V
4 0 V
5 0 V
4 0 V
4 1 V
5 0 V
4 0 V
4 0 V
4 0 V
5 1 V
4 0 V
4 0 V
stroke 5737 4496 M
4 0 V
5 0 V
4 1 V
4 0 V
5 0 V
4 0 V
4 0 V
4 1 V
5 0 V
4 0 V
4 0 V
4 0 V
5 1 V
4 0 V
4 0 V
4 0 V
5 0 V
4 1 V
4 0 V
5 0 V
4 0 V
4 0 V
4 1 V
5 0 V
4 0 V
4 0 V
4 0 V
5 0 V
4 1 V
4 0 V
5 0 V
4 0 V
4 0 V
4 1 V
5 0 V
4 0 V
4 0 V
4 0 V
5 1 V
4 0 V
4 0 V
5 0 V
4 0 V
4 0 V
4 1 V
5 0 V
4 0 V
4 0 V
4 0 V
5 1 V
4 0 V
4 0 V
5 0 V
4 0 V
4 0 V
4 1 V
5 0 V
4 0 V
4 0 V
4 0 V
5 1 V
4 0 V
4 0 V
4 0 V
5 0 V
4 0 V
4 1 V
5 0 V
4 0 V
4 0 V
4 0 V
5 0 V
4 1 V
4 0 V
4 0 V
5 0 V
4 0 V
4 1 V
5 0 V
4 0 V
4 0 V
4 0 V
5 0 V
4 1 V
4 0 V
4 0 V
5 0 V
4 0 V
4 0 V
5 1 V
4 0 V
4 0 V
4 0 V
5 0 V
4 0 V
4 1 V
4 0 V
5 0 V
4 0 V
4 0 V
4 0 V
5 1 V
4 0 V
4 0 V
stroke 6181 4515 M
5 0 V
4 0 V
4 0 V
4 1 V
5 0 V
4 0 V
4 0 V
4 0 V
5 0 V
4 1 V
stroke
2.000 UL
LTb
LCb setrgbcolor
1.000 UL
LTb
LCb setrgbcolor
860 4799 N
860 640 L
5979 0 V
0 4159 V
-5979 0 V
Z stroke
1.000 UP
1.000 UL
LTb
LCb setrgbcolor
stroke
grestore
end
showpage
  }}%
  \put(5936,803){\makebox(0,0)[r]{\strut{}math.h}}%
  \put(5936,1003){\makebox(0,0)[r]{\strut{}My arctan}}%
  \put(3849,140){\makebox(0,0){\strut{}$x$}}%
  \put(160,2719){%
  \special{ps: gsave currentpoint currentpoint translate
630 rotate neg exch neg exch translate}%
  \makebox(0,0){\strut{}$y$}%
  \special{ps: currentpoint grestore moveto}%
  }%
  \put(6839,440){\makebox(0,0){\strut{}$14$}}%
  \put(5985,440){\makebox(0,0){\strut{}$12$}}%
  \put(5131,440){\makebox(0,0){\strut{}$10$}}%
  \put(4277,440){\makebox(0,0){\strut{}$8$}}%
  \put(3422,440){\makebox(0,0){\strut{}$6$}}%
  \put(2568,440){\makebox(0,0){\strut{}$4$}}%
  \put(1714,440){\makebox(0,0){\strut{}$2$}}%
  \put(860,440){\makebox(0,0){\strut{}$0$}}%
  \put(740,4799){\makebox(0,0)[r]{\strut{}$1.6$}}%
  \put(740,4279){\makebox(0,0)[r]{\strut{}$1.4$}}%
  \put(740,3759){\makebox(0,0)[r]{\strut{}$1.2$}}%
  \put(740,3239){\makebox(0,0)[r]{\strut{}$1$}}%
  \put(740,2720){\makebox(0,0)[r]{\strut{}$0.8$}}%
  \put(740,2200){\makebox(0,0)[r]{\strut{}$0.6$}}%
  \put(740,1680){\makebox(0,0)[r]{\strut{}$0.4$}}%
  \put(740,1160){\makebox(0,0)[r]{\strut{}$0.2$}}%
  \put(740,640){\makebox(0,0)[r]{\strut{}$0$}}%
\end{picture}%
\endgroup
\endinput

  \caption{A plot of the error function calculated from the differential equation.}
  \label{fig:erf}
\end{figure}
\end{document}
